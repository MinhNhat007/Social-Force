\chapter{Formulacja modelu 'Social force'}
\section{Zbudowanie ogólne równanie}
\hspace{4ex}Zgodnie z koncepcją 'Social force' przez {\it Helbing \& Molnar 1995} my możemy uznać, że ruch pieszego można opisać za pomocą trzech różnych składników tak, że 
$$f_{i}^{o}$$  \centerline{wewnętrzne przyspieszenie zachowanie, odzwierciedlając motywację pieszego} \centerline{do poruszania się w określonym kierunku z określoną prędkością.} 
$$f_{i}^{wall}$$ \centerline{wpływ ścian korytarza na tego pieszego}
$$f_{ij}$$ \centerline{efekty interakcji odzwierciedlające reakcję pieszych j do innego pieszego i.}
\par \medskip W tym momencie, poznamy zmianą prędkości $v_{i}$ pieszego i coraz możemy wypisać takie równanie w formularze
$$
\frac{dv_{i}}{dt} = f_{i}^{o} + f_{i}^{wall} + f_{ij}
$$
\section{Zbudowanie komponentów}
\hspace{4ex}Łatwo widzimy, że wykorzystujemy dane eksperymentalne do sprawdzenia poprawności powyższego równania i określenia najważniejszej funkcji interakcji $f_{ij}$. Przejdźmy przez wszystkie komponenty.
\subsection{Zachowanie pojedynczego pieszego}
\hspace{4ex}Zgodnie z koncepcją 'Social force' przez {\it Helbing \& Molnar 1995} my mamy równanie dla wewnętrznego przyspieszenia zachowania $\vec{f_i^o}$:
$$\vec{f_{i}^{o}} = \frac{v_i^oe_i^o-v_{i}(t)}{\tau}$$
Gdy,\\ \centerline{$v_i^o = 1.29 \pm 0.19(ms^{-1})$ : pożądane prędkości}
\centerline{$v_i(t) (ms^{-1})$ : aktualna prękość}
\centerline{$\tau = 0.54 \pm 0.05(s)$: czas relaksacji}
\centerline{$e_i^o$: pożądany kierunek ruchu}
\subsection{Wpływ ścian na tym pieszym}
\hspace{4ex}Zgodnie z wcześniejszymi ustaleniami przez {\it Johansson et al. 2007} takie efekty ścian korytarzy na tym pieszym można opisać za pomocą równania:
$$
f_i^{wall}(d_w) = ae^{\frac{-d_w}{b}}
$$
Gdy, \\
\centerline{$d_w (m)$ : odległość prostopadła z pieszego do ściany}
\centerline{$a = 3$ i $b = 0.1$ : parametry odpowiadające do siły odpychania tego samego rzędu}
\subsection{Interakcje międzyludzkie}
\hspace{4ex}Również zgodnie z poprzednego badania poprzez {\it Johansson et al. 2007}, my też wiedzieliśmy, że interakcje międzyludzkie $f_{ij}$ mogą być definiować interakcje międzyludzkie jako funkcja odległości i kąta podejścia okazują się jasne i uzasadnione $f_{ij}(d,\theta)$ i mamy równanie dla tej funkcji
\newpage
$$
f_{ij}(d,\theta) = -Ae^{\frac{-d}{B}}(e^{-(n'B\theta)^2}t + e^{-(nB\theta)^2}n)
$$
Gdy,\\
\centerline{$n_{ij}$ : zmiany kierunkowe jako wektor jednostkowy w lewym stronie $\vec{t_{ij}}$}
\centerline{$d(m)$ : odległość między dwoma pieszymi $i$'em i $j$'em}
\centerline{$\theta(rad)$ : kąt między kierunkiem interakcji a wektorem skierowanym od pieszego $i$ do $j$}
\centerline{$A = 4.5 \pm 0.3$}
\centerline{$n' = 2.0 \pm 0.1$}
\centerline{$n = 3.0 \pm 0.7$}
\centerline{$t_{ij} = \frac{\vec{D_{ij}}}{\Vert \vec{D_{ij}} \Vert}$ : kierunek interakcji}\\
\par
Przyjrzyjmy się bliżej parametrowi B: jest zwiększany w kierunku interakcji przez duże prędkości względne i jest zmniejszone gdy odpychanie w kierunku boków. Więc it depends on:
$$
B = \gamma \Vert D \Vert
$$
Gdy,\\
\centerline{$\gamma = 0.35 \pm 0.01$ : parameter równania}
I mamy \\ \centerline{$\vec{D_{ij}} = \lambda(\vec{v_i} - \vec{v_j}) + \vec{e_{ij}}$}
\centerline{$\lambda = 2.0 \pm 0.2$ : względne znaczenie dwóch kierunków}
\centerline{$\vec{e_{ij}} = \frac{\vec{x_j}-\vec{x_i}}{\Vert \vec{x_j} - \vec{x_i} \Vert}$ : kierunek, w którym znajduje się pieszy $j$}
