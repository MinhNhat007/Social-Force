\chapter{Wprowadzenie i cel projektu}
\hspace{4ex}W ciągu ostatnich dwóch dekad modele opisujące ruch pieszych znalazły znaczące zainteresowanie. Maja one bowiem ogromne znaczenie podczas projektowania i planowania stref masowego użytku publicznego, np. metra lub stacji kolejowych, stadionów, centr handlowych, dzięki możliwości zasymulowania ewakuacji czy innych zjawisk jak problem wąskiego gardła, rozładowania tłumu.

Wszystkie wielkości modelowe, takie jak współrzędne $\vec{r}$ i prędkość $\vec{v}$ pieszych, są możliwe do zmierzenia, a zatem dane symulacyjne są porównywalne z danymi eksperymentalnymi, co daje łatwość w kalibracji modelu symulacyjnego. 
\par \medskip
W przeszłości wiele osób podejmowało projekty badawcze związane z opracowanie modelu najlepiej opisującego ruch pieszych np. 
\textit{Couzin \& Krause 2003; Ball 2004; Sumpter 2006; Helbing \& Molnar 1995; itd.}. 

W następnym paragrafie przedstawimy model "Social Force" dla ruchu pieszych zaproponowany pierwotnie w 1995 roku przez Helbinga i Molnara. Model ten opiera się na koncepcji "Social Force" jako siły oddziałującej na pieszego pochodzącej od interakcji z innymi pieszymi w tłumie czy interakcji otoczeniem np. ścianami.

W jaki sposób piesi modyfikują swoje zachowanie w odpowiedzi na interakcje z innymi? Odpowiedź na to pytanie pozwala zrozumieć mechanizmy prowadzące do samoorganizacji w tłumie i pomaga budować niezawodne modele.
