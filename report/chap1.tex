\chapter{Wprowadzenie i celem projektu}
\hspace{4ex}W ciągu ostatnich dwóch dekad, modele traktowań pieszych znalazły znaczące zainteresowanie z kilku powodów. Po pierwsze, takie modele istnieją pewne analogie z gazami i płynami. Po drugie, wszystkie wielkości modelowe, takie jak współrzędne $\vec{r}$ i prędkość $\vec{v}$ pieszych, są możliwe mierzalne, a zatem testowalne z danymi eksperymentymi. Po trzecie, modele dla pieszych pewno mają ogromną wartości do projektowania i planowania stref dla pieszych, metra lub stacje kolejowe, duże budynki, centra handlowe, itd.
\par \medskip
W przeszłości wiele osób implementuje takie projekty badawcze np. 
\textit{Couzin \& Krause 2003; Ball 2004; Sumpter 2006; Helbing \& Molnar 1995; itd.}. W następującym, przedstawimy model 'Social Force' dla ruchu pieszego a to ozanacza implementujemy projekt poprzez pomiar i modelowanie interakcji między pieszymi.. Więc w jaki sposób pieszy modyfikuje zachowanie w odpowiedzi na interakcje z innymi pieszymi? Odpowiedź na to pytanie pozwala zrozumieć mechanizmy prowadzące do samoorganizacji w 'crowd' i pomaga budować niezawodne modele 'crowd'.
