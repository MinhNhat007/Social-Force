\chapter{Koncepcja 'Social force'}
\hspace{4ex}Wiele osób ma poczucie, że ludzkie zachowania są {\it chaotyczne} lub przynajmniej bardzo nieregularne i nieprzewidywalne. W rzeczywistości, szczególnie w dużych zbiorowiskach, ludzkie zachowanie można opisać jako model matematyczny, a w związku z tym da się przewidzieć zachowanie ludzi.
\par \medskip
Sugeruje się, że ruch pieszych można opisać tak, jak gdyby podlegał 'Social Force'. Te siły nie są bezpośrednio wywierane przez środowisko zewnętrze pieszego, ale odzwierciedlają one wewnętrzne motywacje pieszego do wykonywania określonych czynności jako odpowiedź na interakcje ze środowiskiem zewnętrznym. W prezentowanym modelu zachowań pieszych zasadnicze znaczenie ma kilka sił. Po pierwsze, konstrukcja opisująca siłę związaną z oczywistym dążeniem pieszego do wyjścia czy przejścia. Po drugie, konstrukcja odzwierciedlająca wpływ innych pieszych, między innymi związana z zachowaniem pewnej odległości pomiędzy pieszymi, tzw. strefy komfortu, a także umożliwiającej pieszemu wykonanie kolejnego kroku. Po trzecie, konstrukcja będąca odzwierciedleniem wpływu wszelkiego rodzaju przeszkód np. ścian. Często dodaje się również składową odzwierciedlającą siłę przykuwania uwagi pieszego do czegoś interesującego, a także siłę odzwierciedlającą to, że często piesi poruszają się w grupach znajomych.

Komputerowe symulacje ruchu oddziałujących na siebie pieszych pokazują, że model 'Social force' jest bardzo realistycznie.

