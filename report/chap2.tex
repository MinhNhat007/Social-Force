\chapter{Koncepcja 'Social force'}
\hspace{4ex}Wiele osób ma poczucie, że ludzkie zachowania są {\it chaotyczne} lub przynajmniej bardzo nieregularne i nieprzewidywalne i najprawdopodobnie w przypadku zachowań występują w złożonych sytuacjach. W realistycznym, szczególnie w ogromnej populacji jednostek, ludzkie zachowanie można opisać jako model i można go przewidzieć. Ten punkt widzenia jest podstawową ideą stworzenia koncepcji 'Social force' ruchu pieszych.
\par \medskip
Sugeruje się, że ruch pieszych można opisać tak, jak gdyby podlegałyby 'Social Force'. Te siły nie są bezpośrednio wywierane przez środowisko pieszych, ale te siły są pomiary wewnętrznego motywacje do wykonywania określonych czynności. W prezentowanym modelu zachowań pieszych zasadnicze znaczenie ma kilka wymuszeń siłowych: Po pierwsze, koncepcja opisująca przyspieszenie w kierunku pożądanej prędkości ruchu. Po drugie, koncepcja odzwierciedlająca, że pieszy utrzymuje pewną odległość od innych pieszych i granicach. Komputerowe symulacje 'crowd' oddziałujących na siebie pieszych pokazują, że model 'Social force' jest bardzo realistycznie.

