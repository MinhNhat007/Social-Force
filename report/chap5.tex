\chapter{Wnioski}
\section{Zebranie najważniejszych wniosków}
\hspace{4ex}Opracowaliśmy narzędzie programowe zdolne do symulowania i renderowania ruchu dziesiątek a nawet setek pieszych w czasie rzeczywistym. Model Social Force został zaimplementowany w języku C++ z wykorzystaniem zewnętrznych bibliotek glut.h oraz Qt do obsługi interfejsu. Symulacja pokazała, że ruch pieszych można opisać za pomocą prostego modelu siły społecznej dla indywidualnych zachowań pieszych. Pomimo uproszczonego modelu, w zadowalającym stopniu odzwierciedla on rzeczywiste zjawiska zachodzące w ruchu pieszych.

\section{Zebranie najważniejszych wyzwań i trudności rozpatrywanego problemu}
\hspace{4ex}Wystąpiło wiele trudności w realizacji projektu, ale na szczęście znaleźliśmy rozwiązanie (choć nie zawsze idealne). Mieliśmy problem z automatycznym znajdywaniem \emph{waypoint'ów}. Z powodu ograniczonego czasy realizacji projektu, nie udało nam się zaimplementować automatycznego wykrywania takich punktów (np. z wykorzystaniem algorytmu Dijkstry), dlatego przyjęliśmy ręczne ustalanie takich punktów. Innym problemem jest złożoność obliczeniowa, która przy dużej liczbie aktorów powoduje znaczące spowolnienie działania symulacji. Planowaliśmy wykorzystanie wielowątkowości, jednak również i w tym przypadku nie wystarczyło nam wyznaczonego czasu.
\section{Future Works}
\hspace{4ex}Planujemy obecnie zaimplementować algorytmy bardziej skomplikowane w jeszcze bardziej skomplikowanej mapie. Planujemy np. dodanie tzw. \emph{attracting force} jako losowej siły przyciągającej uwagę pieszych czy zaimplementowanie algorytmu Dijkstry jako wybieranie przez pieszego automatycznie najlepszej ścieżki. Dodatkowo chcemy wykorzystać możliwości \emph{multithreading}.  Jeśli się nam powiedzie, przyszłym krokiem być może zastosowanie Social Force do opisu procesu formułowania opinii, dynamiki grupy lub innych zjawisk społecznych.